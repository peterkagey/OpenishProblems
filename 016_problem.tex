\documentclass{article}

\usepackage[margin=1in]{geometry}
\usepackage{amsmath,amsthm,amssymb}
\usepackage{bbm,enumerate,mathtools}
\usepackage{tikz}

\newenvironment{question}{\begin{trivlist}\item[\textbf{Question.}]}{\end{trivlist}}
\newenvironment{note}{\begin{trivlist}\item[\textbf{Note.}]}{\end{trivlist}}
\newenvironment{references}{\begin{trivlist}\item[\textbf{References.}]}{\end{trivlist}}
\newenvironment{related}{\begin{trivlist}\item[\textbf{Related.}]\end{trivlist}\begin{enumerate}}{\end{enumerate}}
\begin{document}

\title{Problem 16.}
\date{}
\author{Peter Kagey}
\maketitle
  Richard Guy beat me to this problem by a few years. (https://arxiv.org/abs/1207.5099).\\
  John Conway described the ``Subprime Fibonacci Sequence'': \[
    a(1) = a, a(2) = b, a(n + 1) = \operatorname{gpd}(a(n) + a(n - 1)),
  \] where $\operatorname{gpd}(k)$ is the greatest proper divisor of $k$.\\
  Conway then conjectured that regardless of the starting terms, the sequence
  ends in a handful of cycles. Richard Guy found that there are more cycles than
  those that Conway conjectured.

\begin{question}
  What are all of the different possible end behaviors of
  Conway's Subprime Fibonacci Sequence?
\end{question}

\end{document}
