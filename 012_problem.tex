\documentclass{article}

\usepackage[margin=1in]{geometry}
\usepackage{amsmath,amsthm,amssymb}
\usepackage{bbm,enumerate,mathtools}
\usepackage{tikz}

\newenvironment{question}{\begin{trivlist}\item[\textbf{Question.}]}{\end{trivlist}}
\newenvironment{note}{\begin{trivlist}\item[\textbf{Note.}]}{\end{trivlist}}
\newenvironment{references}{\begin{trivlist}\item[\textbf{References.}]}{\end{trivlist}}
\newenvironment{related}{\begin{trivlist}\item[\textbf{Related.}]\end{trivlist}\begin{enumerate}}{\end{enumerate}}
\begin{document}

\title{Problem 12.}
\date{}
\author{Peter Kagey}
\maketitle
  Consider Ron Graham's sequence for $\operatorname{lcm}$, that is, look at sequences such that \[
    n = a_1 < a_2 < \hdots < a_T = k \text{ and } \operatorname{lcm}(a_1,\hdots,a_T) \text{ is square.}
  \]
\begin{question}
  What is the least k (as a function of n) such that such a sequence exists?
\end{question}
\begin{figure}[!h]
  \centering
  \begin{alignat*}{2}
  a(1) &= 1  &&\text{ via } (1)\\
  a(2) &= 4  &&\text{ via } (2, 4)\\
  a(3) &= 3  &&\text{ via } (3, 9)\\
  a(4) &= 4  &&\text{ via } (4)\\
  a(5) &= 25 &&\text{ via } (5, 25)\\
  a(6) &= 12 &&\text{ via } (6, 9, 12)\\
  a(7) &= 49 &&\text{ via } (7, 49)\\
  a(8) &= 16 &&\text{ via } (8, 16)
  \end{alignat*}
  \caption{
    Examples of $a(n)$ for $n \in \{1, 2,\hdots,8\}$.
  }
\end{figure}

\begin{related}
  \item For what values $n$ is $a(n)$ nonsquare?
  \item For what values $n$ does the corresponding sequence have three or more terms?
  \item What is the analogous sequence for perfect cubes, etc?
\end{related}

\end{document}
