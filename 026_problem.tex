\documentclass{article}
\usepackage[margin=1in]{geometry}
\usepackage{amsmath,amsthm,amssymb}
\usepackage{bbm,enumerate,mathtools}
\usepackage{tikz,pgfplots}
\usepackage{chessboard}
\usepackage[hidelinks]{hyperref}
\usepackage{multicol} % Problem 35

\newenvironment{question}{\begin{trivlist}\item[\textbf{Question.}]}{\end{trivlist}}
\newenvironment{note}{\begin{trivlist}\item[\textbf{Note.}]}{\end{trivlist}}
\newenvironment{references}{\begin{trivlist}\item[\textbf{References.}]}{\end{trivlist}}
\newenvironment{related}{\begin{trivlist}\item[\textbf{Related.}]\end{trivlist}\begin{enumerate}}{\end{enumerate}}


\begin{document}
\def\firstlist{{2,3,4,5,6,7,8,9,10}}
\def\secondlist{{2,5,2,5,6,7,8,9,10}}
\def\thirdlist{{2,5,2,7,3,7,8,9,10}}
\def\fourthlist{{2,5,2,7,3,9,4,9,10}}
\def\fifthlist{{2,5,2,7,6,3,4,9,10}}
\def\sixthlist{{2,5,2,9,2,3,4,9,10}}
\def\seventhlist{{2,5,5,3,2,3,4,9,10}}
\def\eighthlist{{2,5,5,3,2,3,7,3,10}}

The prime ant looks along the number line starting at $2$. When she reaches a
composite number, she divides by its least prime factor, and adds that factor to
the previous term, and steps back.

\begin{figure}[!h]
  \centering
  \begin{tikzpicture}
    \foreach \x [evaluate=\x as \nodename using ({\firstlist[\x]})] in {0,...,8} {
      \node at (\x, 0) {\nodename};
    }
    \draw[thick, red] (2, -0) circle (0.3cm);

    \foreach \x [evaluate=\x as \nodename using ({\secondlist[\x]})] in {0,...,8} {
      \node at (\x, -1) {\nodename};
    }
    \draw[thick, red] (4, -1) circle (0.3cm);

    \foreach \x [evaluate=\x as \nodename using ({\thirdlist[\x]})] in {0,...,8} {
      \node at (\x, -2) {\nodename};
    }
    \draw[thick, red] (6, -2) circle (0.3cm);

    \foreach \x [evaluate=\x as \nodename using ({\fourthlist[\x]})] in {0,...,8} {
      \node at (\x, -3) {\nodename};
    }
    \draw[red, thick] (5, -3) circle (0.3cm);

    \foreach \x [evaluate=\x as \nodename using ({\fifthlist[\x]})] in {0,...,8} {
      \node at (\x, -4) {\nodename};
    }
    \draw[red, thick] (4, -4) circle (0.3cm);

    \foreach \x [evaluate=\x as \nodename using ({\sixthlist[\x]})] in {0,...,8} {
      \node at (\x, -5) {\nodename};
    }
    \draw[red, thick] (3, -5) circle (0.3cm);

    \foreach \x [evaluate=\x as \nodename using ({\seventhlist[\x]})] in {0,...,8} {
      \node at (\x, -6) {\nodename};
    }
    \draw[red, thick] (7, -6) circle (0.3cm);

    \foreach \x [evaluate=\x as \nodename using ({\eighthlist[\x]})] in {0,...,8} {
      \node at (\x, -7) {\nodename};
    }
    \draw[red, thick] (8, -7) circle (0.3cm);
  \end{tikzpicture}
  \caption{An illustration of the prime ant's positions after the first 7 steps.}
\end{figure}

\begin{question}
  Does the ant eventually stay to the right of any fixed position?
\end{question}
\begin{related}
  \item Are there any positions that stay permanently greater than $7$? Than $11$?
  \item Does sequence of numbers converge in the long run? If so, what to?
    $(2, 5, 5, 3, 2, \hdots)$
  \item Let $S$ be a subset of $\mathbb{N}$ and let
    $f: S\times S^{c} \rightarrow \mathbb{N}^2$.
    For what ``interesting'' sets $S$ and functions $f$ can we answer the above
    questions?\\
    (In the example $S$ is the prime numbers and $f$ maps
    $(p, c) \mapsto (p + \operatorname{lpf}(c), \operatorname{gpf}(c))$.)
\end{related}
\begin{references}
  \item https://codegolf.stackexchange.com/q/144695/53884
  \item https://math.stackexchange.com/q/2487116/121988
  \item https://oeis.org/A293689
\end{references}
\end{document}
