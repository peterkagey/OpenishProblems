\documentclass{article}
\usepackage[margin=1in]{geometry}
\usepackage{amsmath,amsthm,amssymb}
\usepackage{bbm,enumerate,mathtools}
\usepackage{tikz,pgfplots}
\usepackage{chessboard}
\usepackage[hidelinks]{hyperref}
\usepackage{multicol} % Problem 35

\newenvironment{question}{\begin{trivlist}\item[\textbf{Question.}]}{\end{trivlist}}
\newenvironment{note}{\begin{trivlist}\item[\textbf{Note.}]}{\end{trivlist}}
\newenvironment{references}{\begin{trivlist}\item[\textbf{References.}]}{\end{trivlist}}
\newenvironment{related}{\begin{trivlist}\item[\textbf{Related.}]\end{trivlist}\begin{enumerate}}{\end{enumerate}}


\begin{document}
  Ron Graham's Sequence (A006255) is the least k for which there exists a
  strictly increasing sequence \[
    n = b_1 < b_2 < \hdots < b_t = k \text{ where }
    b_1 \cdot\hdots\cdot b_t \text{ is square.}
  \]
  There is a known way to efficiently compute analogous functions $a_p$ where
  $a_p(n)$ is the least integer such that there exists a sequence
  \begin{enumerate}[(a)]
    \item $n = b_1 \leq b_2 \leq \hdots \leq b_t = a_p(n)$,
    \item any term appears at most $p - 1$ times, and
    \item $b_1 \cdot b_2 \cdot\hdots\cdot b_t$ is a $p$-th power.
  \end{enumerate}
\begin{question}
  An efficient way to compute $a_p$ is known when $p$ is prime.
  What is an efficient way to compute $a_c$ when $c$ is composite?
\end{question}
\begin{figure}[!h]
  \centering
  \begin{alignat*}{3}
  a_4(1) &= 1      &&\text{ via } 1                                     &&= 1^4\\
  a_4(2) &= 2      &&\text{ via } 2^2 \cdot 4                           &&= 2^4\\
  a_4(3) &= 6      &&\text{ via } 3^2 \cdot 4 \cdot 6^2                 &&= 6^4\\
  a_4(4) &= 4      &&\text{ via } 4^2                                   &&= 2^4\\
  a_4(5) &= 10     &&\text{ via } 5^2 \cdot 8^2 \cdot 10^2              &&= 20^4\\
  a_4(6) &= 9      &&\text{ via } 6^2 \cdot 8^2 \cdot 9                 &&= 12^4\\
  a_4(7) &= 14     &&\text{ via } 7^2 \cdot 8^2 \cdot 14^2              &&= 28^4\\
  a_4(8) &\leq 15  &&\text{ via } 8^2 \cdot 9 \cdot 10^2 \cdot 15^2     &&= 60^4\\
  a_4(9) &= 9      &&\text{ via } 9^2                                   &&= 3^4\\
  a_4(10) &\leq 18 &&\text{ via } 10^2 \cdot 12^2 \cdot 15^2 \cdot 18^2 &&= 180^4
  \end{alignat*}
  \caption{
    Examples of $a_4(n)$ for $n \in \{1, 2,\hdots,10\}$.
  }
\end{figure}

\begin{related}
  \item For what values $n$ is $a_4(n) < A006255(n)$?
  \item Given some integers $k, c$, how many terms have $a_c(n) = k$?
    (e.g. $a_4(6) = a_4(9) = 9$.)
  \item Does $a_c$ contain arbitrarily many copies of the same value? \\
    (i.e. does there exist a sequence such that
    $a_4(n_1) = a_4(n_2) = \hdots = a_4(n_m)$ for arbitrarily large $m$?)
  \item How many times does $k$ appear in the image of $a_c$?
    (e.g. $9$ appears twice, as $a(6)$ and $a(9)$.)
  \item What integers are in the image of $a_c$?
\end{related}

\begin{references}
  \item \url{https://oeis.org/A300516}
\end{references}
\end{document}
