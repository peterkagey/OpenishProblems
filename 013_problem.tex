\documentclass{article}

\usepackage[margin=1in]{geometry}
\usepackage{amsmath,amsthm,amssymb}
\usepackage{bbm,enumerate,mathtools}
\usepackage{tikz}

\newenvironment{question}{\begin{trivlist}\item[\textbf{Question.}]}{\end{trivlist}}
\newenvironment{note}{\begin{trivlist}\item[\textbf{Note.}]}{\end{trivlist}}
\newenvironment{references}{\begin{trivlist}\item[\textbf{References.}]}{\end{trivlist}}
\newenvironment{related}{\begin{trivlist}\item[\textbf{Related.}]\end{trivlist}\begin{enumerate}}{\end{enumerate}}
\begin{document}

\title{Problem 13.}
\date{}
\author{Peter Kagey}
\maketitle
  Ron Graham's (A006255) sequence is the least k for which there exists a
  strictly increasing sequence \[
    n = a_1 \leq a_2 \leq \hdots \leq a_T = k \text{ where }
    a_1 \cdot\hdots\cdot a_T \text{ is square.}
  \] and where any term appears at most $p - 1$ times.
  There is a known way to efficiently compute analogous sequences wherein
  $a_1 \cdot\hdots\cdot a_T$ is a $p$-th power, where $p$ is a prime.
\begin{question}
  What is an efficient way to compute analogous sequences wherein
  $a_1 \cdot\hdots\cdot a_T$ is a $c$-th power, where $c$ is composite?
\end{question}
\begin{figure}[!h]
  \centering
  \begin{alignat*}{3}
  a_4(1) &= 1      &&\text{ via } 1                                     &&= 1^4\\
  a_4(2) &= 2      &&\text{ via } 2^2 \cdot 4                           &&= 2^4\\
  a_4(3) &= 6      &&\text{ via } 3^2 \cdot 4 \cdot 6^2                 &&= 6^4\\
  a_4(4) &= 4      &&\text{ via } 4^2                                   &&= 2^4\\
  a_4(5) &= 10     &&\text{ via } 5^2 \cdot 8^2 \cdot 10^2              &&= 20^4\\
  a_4(6) &= 9      &&\text{ via } 6^2 \cdot 8^2 \cdot 9                 &&= 12^4\\
  a_4(7) &= 14     &&\text{ via } 7^2 \cdot 8^2 \cdot 14^2              &&= 28^4\\
  a_4(8) &\leq 15  &&\text{ via } 8^2 \cdot 9 \cdot 10^2 \cdot 15^2     &&= 60^4\\
  a_4(9) &= 9      &&\text{ via } 9^2                                   &&= 3^4\\
  a_4(10) &\leq 18 &&\text{ via } 10^2 \cdot 12^2 \cdot 15^2 \cdot 18^2 &&= 180^4
  \end{alignat*}
  \caption{
    Examples of $a_4(n)$ for $n \in \{1, 2,\hdots,10\}$.
  }
\end{figure}

\begin{related}
  \item For what values $n$ is $a(n) < A006255(n)$?
  \item How many $c$-th power sequences have $a_T = a_c(n)$?
  \item Do any such $c$-th power sequences exactly two distinct terms?
\end{related}

\end{document}
