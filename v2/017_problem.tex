\documentclass{article}
\usepackage[margin=1in]{geometry}
\usepackage{amsmath,amsthm,amssymb}
\usepackage{bbm,enumerate,mathtools}
\usepackage{tikz,pgfplots}
\usepackage{chessboard}
\usepackage[hidelinks]{hyperref}
\usepackage{multicol} % Problem 35

\newenvironment{question}{\begin{trivlist}\item[\textbf{Question.}]}{\end{trivlist}}
\newenvironment{note}{\begin{trivlist}\item[\textbf{Note.}]}{\end{trivlist}}
\newenvironment{references}{\begin{trivlist}\item[\textbf{References.}]}{\end{trivlist}}
\newenvironment{related}{\begin{trivlist}\item[\textbf{Related.}]\end{trivlist}\begin{enumerate}}{\end{enumerate}}



\begin{document}
\rating{3}{3}
Consider placing any number of queens (of the same color) on an $n \times n$
chessboard in such a way as to maximize the number of legal moves available.
\begin{figure}[!h]
  \centering
  \chessboard[maxfield=c3,setwhite={Qa3,Qa2,Qb1,Qc3},showmover=false]
  \chessboard[maxfield=d4,setwhite={Qa1,Qa3,Qa4,Qb1,Qc4,Qd1,Qd2,Qd4},showmover=false]
  \chessboard[maxfield=e5,setwhite={Qa2,Qa4,Qb1,Qb5,Qc1,Qc3,Qc5,Qd1,Qd5,Qe2,Qe4},showmover=false]

  \caption{
    Examples of $a_q(3) = 17, a_q(4) = 40, a_q(5) = 76$.
  }
\end{figure}

\begin{question}
  Is Alec Jones's conjecture true: $a_q(n) = 8(n-2)^2$ for $n \geq 6$, by
  placing the queens around the perimeter?
\end{question}
\begin{related}
  \item What about the analogous function for rooks ($a_r$) or bishops ($a_b$)?
  \item What if the chessboard is a torus? Cylinder? M\"obius strip?
  \item What if the chessboard is $n \times m$?
  \item Is $a_b(n) = \lfloor a_q(n)/2 \rfloor$? for all $n$?
  \item What if queens can attack?
\end{related}

\begin{references}
  \item A278211: \url{http://oeis.org/A278211}
  \item A278212: \url{http://oeis.org/A278212}
  \item A275815: \url{http://oeis.org/A275815}
\end{references}
\end{document}
