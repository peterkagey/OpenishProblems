\documentclass{article}
\usepackage[margin=1in]{geometry}
\usepackage{amsmath,amsthm,amssymb}
\usepackage{bbm,enumerate,mathtools}
\usepackage{tikz,pgfplots}
\usepackage{chessboard}
\usepackage[hidelinks]{hyperref}
\usepackage{multicol} % Problem 35

\newenvironment{question}{\begin{trivlist}\item[\textbf{Question.}]}{\end{trivlist}}
\newenvironment{note}{\begin{trivlist}\item[\textbf{Note.}]}{\end{trivlist}}
\newenvironment{references}{\begin{trivlist}\item[\textbf{References.}]}{\end{trivlist}}
\newenvironment{related}{\begin{trivlist}\item[\textbf{Related.}]\end{trivlist}\begin{enumerate}}{\end{enumerate}}


\begin{document}
\rating{3}{2}
A relation on a group $X$ is a subset $S \subseteq X \times X$. For a given
relation if $(x,y) \in S$, then we say that $x$ is related to $y$ and denote
it by $xRy$.

\noindent
A relation is called ``antitransitive'' if
$(x,y), (y,z) \in S$ implies $(x,z) \not \in S$.
\newcommand\drawBoundary[2]{%
  \draw[thick, rounded corners=10mm, color=#1!50!black, #2]
    ({-3/2*0.25},{0.25*-sqrt(3)/2})--
    ({5/2*0.25 + 1*0.75},{0.25*-sqrt(3)/2})--
    ({1/2},{0.25*3*sqrt(3)/2+0.75*sqrt(3)/2})--
    cycle
  ;
  \node[circle, fill={#1!50!black}] (1) at (0,0) {};
  \node[circle, fill={#1!50!black}] (2) at (1,0) {};
  \node[circle, fill={#1!50!black}] (3) at (1/2, {sqrt(3)/2}) {};
}

\begin{figure}[ht!]
  \centering
  \noindent
  \begin{tikzpicture}[scale=2]
    \drawBoundary{magenta}{thin}
  \end{tikzpicture}
  \begin{tikzpicture}[scale=2]
    \drawBoundary{green}{dotted}
    \draw[-{Latex[length=3mm]}] (3) edge (2);
  \end{tikzpicture}

  \begin{tikzpicture}[scale=2]
    \drawBoundary{cyan}{solid, thin}
    \draw[-{Latex[length=3mm]}, thick]
      (3) edge (1)
      (3) edge (2)
    ;
  \end{tikzpicture}
  \begin{tikzpicture}[scale=2]
    \drawBoundary{cyan}{thin}
    \draw[-{Latex[length=3mm]}, thick]
      (1) edge (2)
      (2) edge (1)
    ;
  \end{tikzpicture}
  \begin{tikzpicture}[scale=2]
    \drawBoundary{cyan}{thin}
    \draw[-{Latex[length=3mm]}, thick]
      (2) edge (3)
      (3) edge (1)
    ;
  \end{tikzpicture}
  \begin{tikzpicture}[scale=2]
    \drawBoundary{cyan}{thin}
    \draw[-{Latex[length=3mm]}, thick]
      (1) edge (3)
      (2) edge (3)
   ;
  \end{tikzpicture}

  \begin{tikzpicture}[scale=2]
    \drawBoundary{orange}{thin}
    \draw[-{Latex[length=3mm]}, thick]
      (2) edge (3)
      (3) edge (1)
      (3) edge (2)
    ;
  \end{tikzpicture}
  \begin{tikzpicture}[scale=2]
    \drawBoundary{orange}{thin}
    \draw[-{Latex[length=3mm]}, thick]
      (1) edge (3)
      (2) edge (3)
      (3) edge (2)
    ;
  \end{tikzpicture}
  \begin{tikzpicture}[scale=2]
    \drawBoundary{orange}{thin}
    \draw[-{Latex[length=3mm]}, thick]
      (1) edge (3)
      (2) edge (1)
      (3) edge (2)
    ;
  \end{tikzpicture}
  \begin{tikzpicture}[scale=2]
    \drawBoundary{blue}{dotted}
    \draw[-{Latex[length=3mm]}, thick]
        (1) edge (3)
        (2) edge (3)
        (3) edge (1)
        (3) edge (2)
      ;
  \end{tikzpicture}
  \caption{
    The ten antitransitive relations on $3$ unlabeled nodes.
    There are
    $1$, $1$, $4$, $3$, and $1$ relations with
    $0$, $1$, $2$, $3$, and $4$ pairs respectively.
  }
\end{figure}
\begin{question}
  What is the asymptotic growth of the number of antitransitive relations as
  a function of the number of (unlabeled) nodes?
\end{question}

\begin{related}
  \item On $n$ labeled nodes?
  \item Given some subset of conditions
    (e.g. reflexive, asymmetric, antitransitive, connex, etc.), what is the
    asymptotic growth?
  \item What's the ratio of the number of, say,
    transitive relations to antitransitive relations as $n \rightarrow \infty$.
  \item How many relations with exactly $k$ pairs?
  \item What's the greatest number of pairs?
  \item With $\ell$ (strongly) connected components?
\end{related}

\begin{references}
  \item Problem 39.
  \item OEIS sequences
    \href{https://oeis.org/A341471}{A341471} and
    \href{https://oeis.org/A341473}{A341473}.
\end{references}
\end{document}
