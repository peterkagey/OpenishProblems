\documentclass{article}
\usepackage[margin=1in]{geometry}
\usepackage{amsmath,amsthm,amssymb}
\usepackage{bbm,enumerate,mathtools}
\usepackage{tikz,pgfplots}
\usepackage{chessboard}
\usepackage[hidelinks]{hyperref}
\usepackage{multicol} % Problem 35

\newenvironment{question}{\begin{trivlist}\item[\textbf{Question.}]}{\end{trivlist}}
\newenvironment{note}{\begin{trivlist}\item[\textbf{Note.}]}{\end{trivlist}}
\newenvironment{references}{\begin{trivlist}\item[\textbf{References.}]}{\end{trivlist}}
\newenvironment{related}{\begin{trivlist}\item[\textbf{Related.}]\end{trivlist}\begin{enumerate}}{\end{enumerate}}

\begin{document}
\rating{2}{4}
We're interested in looking at ``highly composite'' polyominoes, which are $n$-ominoes
that can be tiled by the greatest number of (weakly) smaller polyominoes,
where we only allow one kind of free polyomino in the tiling.

\begin{figure}[ht!]
  \begin{tikzpicture}[scale=0.75]
    \fill[color6, fill opacity = 0.5, draw=black, line width=1] (0,1) -- ++(0,1) -- ++(1,0) -- ++(0,1) -- ++(1,0) -- ++(0,2) -- ++(1,0) -- ++(0,-1) -- ++(1,0) -- ++(0,-3) -- ++(-1,0) -- ++(0,-1) -- ++(-2,0) -- ++(0,1) -- cycle;
  \end{tikzpicture}
  ~
  \begin{tikzpicture}[scale=0.75]
    \fill[color1, fill opacity = 0.5, draw=black, line width=1] (0,1) -- ++(0,1) -- ++(2,0) -- ++(0,-1) -- ++(1,0) -- ++(0,-1) -- ++(-2,0) -- ++(0,1) -- cycle;
    \fill[color5, fill opacity = 0.5, draw=black, line width=1] (1,2) -- ++(0,1) -- ++(2,0) -- ++(0,-1) -- ++(1,0) -- ++(0,-1) -- ++(-2,0) -- ++(0,1) -- cycle;
    \fill[color9, fill opacity = 0.5, draw=black, line width=1] (2,3) -- ++(0,2) -- ++(1,0) -- ++(0,-1) -- ++(1,0) -- ++(0,-2) -- ++(-1,0) -- ++(0,1) -- cycle;
    \draw (0,1) -- ++(0,1) -- ++(1,0) -- ++(0,1) -- ++(1,0) -- ++(0,2) -- ++(1,0) -- ++(0,-1) -- ++(1,0) -- ++(0,-3) -- ++(-1,0) -- ++(0,-1) -- ++(-2,0) -- ++(0,1) -- cycle;
  \end{tikzpicture}
  ~
  \begin{tikzpicture}[scale=0.75]
    \fill[color1,  fill opacity = 0.5, draw=black, line width=1] (2,2) -- ++(0,1) -- ++(2,0) -- ++(0,-2) -- ++(-1,0) -- ++(0,1) -- cycle;
    \fill[color4,  fill opacity = 0.5, draw=black, line width=1] (0,1) -- ++(0,1) -- ++(1,0) -- ++(0,1) -- ++(1,0) -- ++(0,-2) -- cycle;
    \fill[color7,  fill opacity = 0.5, draw=black, line width=1] (1,0) -- ++(0,1) -- ++(1,0) -- ++(0,1) -- ++(1,0) -- ++(0,-2) -- cycle;
    \fill[color10, fill opacity = 0.5, draw=black, line width=1] (2,3) -- ++(0,2) -- ++(1,0) -- ++(0,-1) -- ++(1,0) -- ++(0,-1) -- cycle;
    \draw (0,1) -- ++(0,1) -- ++(1,0) -- ++(0,1) -- ++(1,0) -- ++(0,2) -- ++(1,0) -- ++(0,-1) -- ++(1,0) -- ++(0,-3) -- ++(-1,0) -- ++(0,-1) -- ++(-2,0) -- ++(0,1) -- cycle;
  \end{tikzpicture}
  ~
  \begin{tikzpicture}[scale=0.75]
    \fill[color1,  fill opacity = 0.5, draw=black, line width=1] (1,0) rectangle ++(2,1);
    \fill[color3,  fill opacity = 0.5, draw=black, line width=1] (0,1) rectangle ++(2,1);
    \fill[color5,  fill opacity = 0.5, draw=black, line width=1] (2,1) rectangle ++(2,1);
    \fill[color7,  fill opacity = 0.5, draw=black, line width=1] (1,2) rectangle ++(2,1);
    \fill[color9,  fill opacity = 0.5, draw=black, line width=1] (3,2) rectangle ++(1,2);
    \fill[color11, fill opacity = 0.5, draw=black, line width=1] (2,3) rectangle ++(1,2);
    \draw (0,1) -- ++(0,1) -- ++(1,0) -- ++(0,1) -- ++(1,0) -- ++(0,2) -- ++(1,0) -- ++(0,-1) -- ++(1,0) -- ++(0,-3) -- ++(-1,0) -- ++(0,-1) -- ++(-2,0) -- ++(0,1) -- cycle;
  \end{tikzpicture}
  ~
  \begin{tikzpicture}[scale=0.75]
    \fill[color1,  fill opacity = 0.5, draw=black, line width=1] (1,0) rectangle ++(1,1);
    \fill[color2,  fill opacity = 0.5, draw=black, line width=1] (2,0) rectangle ++(1,1);
    \fill[color3,  fill opacity = 0.5, draw=black, line width=1] (0,1) rectangle ++(1,1);
    \fill[color4,  fill opacity = 0.5, draw=black, line width=1] (1,1) rectangle ++(1,1);
    \fill[color5,  fill opacity = 0.5, draw=black, line width=1] (2,1) rectangle ++(1,1);
    \fill[color6,  fill opacity = 0.5, draw=black, line width=1] (3,1) rectangle ++(1,1);
    \fill[color7,  fill opacity = 0.5, draw=black, line width=1] (1,2) rectangle ++(1,1);
    \fill[color8,  fill opacity = 0.5, draw=black, line width=1] (2,2) rectangle ++(1,1);
    \fill[color9,  fill opacity = 0.5, draw=black, line width=1] (3,2) rectangle ++(1,1);
    \fill[color10, fill opacity = 0.5, draw=black, line width=1] (3,3) rectangle ++(1,1);
    \fill[color11, fill opacity = 0.5, draw=black, line width=1] (2,3) rectangle ++(1,1);
    \fill[color12, fill opacity = 0.5, draw=black, line width=1] (2,4) rectangle ++(1,1);
    \draw (0,1) -- ++(0,1) -- ++(1,0) -- ++(0,1) -- ++(1,0) -- ++(0,2) -- ++(1,0) -- ++(0,-1) -- ++(1,0) -- ++(0,-3) -- ++(-1,0) -- ++(0,-1) -- ++(-2,0) -- ++(0,1) -- cycle;
  \end{tikzpicture}

  \caption{
    Five (all?) polyominoes that can tile the given $12$-omino.
  }
\end{figure}

\begin{question}
  What do the highly composite polyominoes look like? Are they always rectangles?
\end{question}

\begin{related}
  \item What if this is done for polycubes? On other tilings?
  \item What if we count the number of tilings? (i.e. if there's multiple domino tilings, we count them separately.)
  \item What if we look at tilings by fixed polyominoes? One-sided polyominoes?
\end{related}

\begin{references}
  \item \href{https://oeis.org/A342430}{Drake Thomas's prime polyominoes (A342430).}
  \item Problem 77.
\end{references}
\end{document}
