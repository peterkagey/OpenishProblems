\documentclass{article}
\usepackage[margin=1in]{geometry}
\usepackage{amsmath,amsthm,amssymb}
\usepackage{bbm,enumerate,mathtools}
\usepackage{tikz,pgfplots}
\usepackage{chessboard}
\usepackage[hidelinks]{hyperref}
\usepackage{multicol} % Problem 35

\newenvironment{question}{\begin{trivlist}\item[\textbf{Question.}]}{\end{trivlist}}
\newenvironment{note}{\begin{trivlist}\item[\textbf{Note.}]}{\end{trivlist}}
\newenvironment{references}{\begin{trivlist}\item[\textbf{References.}]}{\end{trivlist}}
\newenvironment{related}{\begin{trivlist}\item[\textbf{Related.}]\end{trivlist}\begin{enumerate}}{\end{enumerate}}


\begin{document}
  \thispagestyle{empty} ~

  \vfill
  \begin{quote}
    Be patient towards all that is unsolved in your heart and try to love the
    questions themselves as if they were locked rooms or books written in a very
    foreign language. Don't search for the answers, which could not be given to
    you now because you would not be able to live them. And the point is to live
    everything. Live the questions now. Perhaps then, someday far in the future,
    you will gradually, without even noticing it, live your way into the answer.
    \begin{flushright}
      \small{Rainer Maria Rilke, \textit{Letters to a Young Poet}}
    \end{flushright}
  \end{quote}

  ~

  \begin{quote}
  My love for this teaching and the seriousness with which I take it rests in
  part on my deep reverence for the gravity and the power of questions in human
  life. I think that this is undervalued in a culture that is in love with the
  form of words that is an answer — and the way with words that is an argument.

  But I also find a question to be a mighty form of words, and I have learned a
  few things about questions. I have learned that questions elicit answers in
  their likeness — that answers rise or fall to the questions they meet. We've
  all seen this. We've all experienced it. It's very hard to respond to a
  combative question with anything but a combative answer. It's almost
  impossible to transcend a simplistic question with anything but a simplistic
  answer. But the opposite is also true: it's hard to resist a generous
  question. This is a skill that needs relearning, but I believe that we all
  have it in us to ask questions that invite, that draw forth searching in
  dignity and revelation. There is something redemptive and lifegiving about
  asking a better question.

  And it is a deep truth in science, and also in each of our lives, that we are
  shaped as much by the quality of the questions we're asking at any given point
  as by the answers we have it in us to give. Those moments in our lives when a
  better quality of question rises up in us, stops us in our tracks — those are
  pivot points. Those are moments where discovery and new possibility break in.

  \begin{flushright}
  \small{
    Krista Tippett,
    \textit{On Being},
    ``\href{https://onbeing.org/living-the-questions-transcript/}{Living the Questions}''.
  }
  \end{flushright}
  \end{quote}
  \vfill
\end{document}
