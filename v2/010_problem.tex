\documentclass{article}
\usepackage[margin=1in]{geometry}
\usepackage{amsmath,amsthm,amssymb}
\usepackage{bbm,enumerate,mathtools}
\usepackage{tikz,pgfplots}
\usepackage{chessboard}
\usepackage[hidelinks]{hyperref}
\usepackage{multicol} % Problem 35

\newenvironment{question}{\begin{trivlist}\item[\textbf{Question.}]}{\end{trivlist}}
\newenvironment{note}{\begin{trivlist}\item[\textbf{Note.}]}{\end{trivlist}}
\newenvironment{references}{\begin{trivlist}\item[\textbf{References.}]}{\end{trivlist}}
\newenvironment{related}{\begin{trivlist}\item[\textbf{Related.}]\end{trivlist}\begin{enumerate}}{\end{enumerate}}


\begin{document}
\rating{3}{2}
Consider Ron Graham's sequence for $\operatorname{LCM}$, that is, look at sequences such that \[
  n = b_1 < b_2 < \hdots < b_t = k \text{ and } \operatorname{LCM}(b_1,\hdots,b_t) \text{ is square.}
\]
\begin{question}
  Let $A300516(n)$ be the least k (as a function of n) such that such a sequence
  exists?
\end{question}
\begin{figure}[h]
  \centering
  \allowdisplaybreaks
  \begin{multicols}{3}
    \noindent
    \begin{alignat*}{2}
      a(1)  &= 1  &&\text{ via } (1)\\
      a(2)  &= 4  &&\text{ via } (2, 4)\\
      a(3)  &= 3  &&\text{ via } (3, 9)\\
      a(4)  &= 4  &&\text{ via } (4)\\
      a(5)  &= 25 &&\text{ via } (5, 25)\\
      a(6)  &= 12 &&\text{ via } (6, 9, 12)\\
      a(7)  &= 49 &&\text{ via } (7, 49)\\
      a(8)  &= 16 &&\text{ via } (8, 16)\\
      a(9)  &= 9  &&\text{ via } (9)\\
      a(10) &= 25 &&\text{ via } (10, 16, 25)\\
      a(11) &= 121 &&\text{ via } (11, 121)\\
      a(12) &= 18  &&\text{ via } (12, 18)\\
      a(13) &= 169 &&\text{ via } (13, 169)\\
      a(14) &= 49  &&\text{ via } (14, 16, 49)\\
      a(15) &= 25  &&\text{ via } (15, 16, 18, 25)\\
      a(16) &= 16  &&\text{ via } (16)\\
      a(17) &= 289 &&\text{ via } (17, 289)\\
      a(18) &= 25  &&\text{ via } (18, 20, 25)\\
      a(19) &= 361 &&\text{ via } (19, 361)\\
      a(20) &= 25  &&\text{ via } (20, 25)\\
      a(21) &= 49  &&\text{ via } (21, 36, 49)\\
      a(22) &= 121 &&\text{ via } (22, 64, 121)\\
      a(23) &= 529 &&\text{ via } (23, 529)\\
      a(24) &= 48  &&\text{ via } (24, 36, 48)\\
      a(25) &= 25  &&\text{ via } (25)\\
      a(26) &= 169 &&\text{ via } (26, 64, 169)\\
      a(27) &= 81  &&\text{ via } (27, 81)\\
      a(28) &= 49  &&\text{ via } (28, 49)\\
      a(29) &= 841 &&\text{ via } (29, 841)\\
      a(30) &= 50  &&\text{ via } (30)
    \end{alignat*}
  \end{multicols}
  \caption{
    Examples of $A300516(n)$ for $1 \leq n \leq 30$.
  }
\end{figure}

\begin{related}
  \item For what values $n$ is $A300516(n)$ nonsquare?
  \item For what values $n$ does the corresponding sequence have three or more terms?
  \item What is the analogous sequence for perfect cubes, etc?
\end{related}

\begin{references}
  \item \url{https://oeis.org/A300516}
\end{references}
\end{document}
