\documentclass{article}
\usepackage[margin=1in]{geometry}
\usepackage{amsmath,amsthm,amssymb}
\usepackage{bbm,enumerate,mathtools}
\usepackage{tikz,pgfplots}
\usepackage{chessboard}
\usepackage[hidelinks]{hyperref}
\usepackage{multicol} % Problem 35

\newenvironment{question}{\begin{trivlist}\item[\textbf{Question.}]}{\end{trivlist}}
\newenvironment{note}{\begin{trivlist}\item[\textbf{Note.}]}{\end{trivlist}}
\newenvironment{references}{\begin{trivlist}\item[\textbf{References.}]}{\end{trivlist}}
\newenvironment{related}{\begin{trivlist}\item[\textbf{Related.}]\end{trivlist}\begin{enumerate}}{\end{enumerate}}


\begin{document}
  Start with $n$ piles with a single stone in each pile. If two piles have the
  same number of stones, then any number of stones can be moved between them.
\begin{figure}[!h]
  \centering
  \begin{tikzpicture}[sibling distance=5em, every node/.style = {align=center}]]
    \node {1, 1, 1, 1, 1}
      child { node {2, 1, 1, 1}
        child { node {2, 2, 1}
          child { node {3, 1, 1}
            child { node {3, 2}}
          }
          child { node {4, 1} }
        }
      };
  \end{tikzpicture}
  \caption{
    An illustration of all possible moves for $n = 5$.
  }
\end{figure}

\begin{question}
  What is the greatest number of steps that can occur? Alternatively how many
  ``levels'' are in the tree of possible moves?
\end{question}

\begin{related}
  \item Let $s$ be the total number of distinct states.
    (The example shows that $s(5) = 6$.)
  \item Let $c$ be the total number of states that \textit{cannot} be acheived.
    (In the example, $c(5) = 1$ via the state $(5)$.)
  \item Is $c(p) = 1$ for all primes $p$?
  \item Is $c(n) = 0$ if and only if $n$ is a power of $2$?
  \item Let $\ell$ be the least number of steps to a terminal state.
    (In the example, $\ell(5) = 3$ ending in the state $(4,1)$.)
  \item Let $g$ be the greatest number of steps to a terminal state.
    (In the example, $g(5) = 4$ ending in the state $(3,2)$.)
  \item Let $p$ be the total number of paths.
    (In the example, $p(5) = 2$.)
  \item Let $t$ be the number of distinct \textit{terminal} states.
    (In the example, $t(5) = 2$ with states $(4,1)$ and $(3,2)$.)
\end{related}

\begin{references}
  \item \url{https://oeis.org/A292836}
\end{references}
\end{document}
